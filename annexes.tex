%\section*{Annexes}
%TODO Glossaire
\renewcommand{\contentsname}{Table des annexes}
\tableofcontents
\clearpage
%\addcontentsline{toc}{subsection}{Glossaire}
\section{Glossaire}

\begin{wrapfigure}[5]{i}{3cm}
\qrcode[height=1in]{https://en.wikipedia.org/wiki/Enterprise_resource_planning}
\end{wrapfigure}

\paragraph{ERP : }

Enterprise resource planning  en anglais. Outil informatique permettant de mettre en lien différente partie d'une même entreprise. Cet outil doit assurer l'intégrité des ressources ainsi qu'il doit premettre de faire des audits.

\begin{wrapfigure}[5]{o}{3cm}
\qrcode[height=1in]{https://www.odoo.com/fr_FR/}
\end{wrapfigure}
\paragraph{Odoo : }

Odoo est un logiciel de gestion d'entreprise dont la première version est sortie en 2004. Il comprend plusieurs modules dont la gestion des ventes, des ressources humaines ou de la comptabilité. Si ça version basique est gratuite il existe une version payante incluant un support.

\begin{wrapfigure}[5]{i}{3cm}
\qrcode[height=1in]{https://en.wikipedia.org/wiki/Unified_Modeling_Language}
\end{wrapfigure}
\paragraph{UML : }

Unified Modeling Language en anglaise. Language et suite de diagramme permettant de définir un cahier des charges dans le domaine informatique. L'analyse se fait d'abord du point de vue du client puis déscend au différentes solutions techniques envisagées et au flux d'information.
\clearpage
%\addcontentsline{toc}{subsection}{Matrice SWOT}
\section{Matrice SWOT}
\newcommand{\texta}{Helpful \tiny (to achieve the objective)\par}
\newcommand{\textb}{Harmful \tiny (to achieve the objective)\par}
\newcommand{\textcn}{Internal origin \tiny (product\slash company attributes)\par}
\newcommand{\textdn}{External origin \tiny (environment\slash market attributes)\par}
\newcommand{\nodestrengths}{Strengths\\\begin{itemize}
  \item Application de prise de mesure
  \item Gestion informatisé des commandes
\end{itemize}}
\newcommand{\nodeweakness}{Weakness\\\begin{itemize}
  \item Barière douanière
  \item Mauvaise image de l'industrie textile en Asie
\end{itemize}}
\newcommand{\nodeopportunities}{Opportunities\\\begin{itemize}
  \item Guerre commercial USA/Chine
  \item Emergence de la classe moyenne asiatique
\end{itemize}}
\newcommand{\nodethreats}{Threats\\\begin{itemize}
  \item Instabilité politique thailandaise
\end{itemize}}
\begin{tikzpicture}[
    any/.style={draw,minimum width=8cm,minimum height=8cm,
                 text width=2.5cm,align=left,outer sep=0pt},
    header/.style={any,minimum height=1cm,fill=black!10},
    leftcol/.style={header,rotate=90}
]

\matrix (SWOT) [matrix of nodes,nodes={any,anchor=center},
                column sep=-\pgflinewidth,
                row sep=-\pgflinewidth,
                row 1/.style={nodes=header},
                column 1/.style={nodes=leftcol},
                inner sep=0pt]
{
          & {\texta} & {\textb} \\
{\textcn} & {\nodestrengths} & {\nodeweakness} \\
{\textdn} & {\nodeopportunities} & {\nodethreats} \\
};
\end{tikzpicture}

\clearpage
%\addcontentsline{toc}{subsection}{Chiffre du textile}
\section{Chiffre du textile}
quelques chiffres sur le textile
