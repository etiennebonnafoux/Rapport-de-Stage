\subsection{Analyse de la stratégie Probespoke}
\paragraph{}
La plus value apportée par Probespoke vient de deux facteurs: la simplification apportée aux tailleurs occidentaux et la différence du coup de main d'oeuvre. Plusieurs zones économiques se partagent le sur-mesure. On peut classer ces zones en deux groupes. Le premier est celui des pays qui ont une forte tradition dans ces domaines à savoir l'Angleterre, la France et l'Italie. Ils se spécialisent dans le sur-mesure haut de gamme à des prix atteignant souvent 1000 euros pour un costume complet. Le deuxième groupe est composé des pays ayant développé plus tardivement une industrie textile dans ce domaine. Il est notamment composé de la Chine, la Thailande, l'Europe de l'Est et le Maghreb. Cette zone produit des costumes de basse et moyenne qualités entre 200 euros et 700 euros. Les équipes sont néanmoins souvent formés par des maitre-tailleurs européens.  Les principaux importateurs sont les États-Unis, l'Europe et plus récemment la classe moyenne asiatique qui s'enrichit.
\paragraph{} Quels sont les aventages (et les incovénients) de s'implenter en Thailande par rapport aux autres pays du même groupe? En Thailande la main d'oeuvre est moins chère qu'au Maghreb et en Europe de l'Est mais plus qu'en Chine. Cependant le dévellopemment du sur-mesure en Chine est plus récent et leur organisation n'est pas encore mature. Ainsi de nombreuses étapes de la confection sont sous-traité ce qui augmente les erreurs de communications. Ainsi de nombreux revendeurs attirées tout d'abord par les prix chinois reviennent ensuite en Thailande. Ce mouvement s'est accélérer cet été, en effet le désaccord commerciale sino-américain pousse de nombreux taileurs américains à quitter la Chine. L'Europe est un marché plus difficile à atteindre; en effet les taxes d'import rendent l'Europe de l'Est et le Maghreb (qui profite d'accord commérciaux) compétitifs devant les prix thailandais. Finalement un atout de la Thailande est son industrie du divertissement. En effet dans de nombreuses entreprises thailandaises il existe un entertaiment officier chargé d'ammener les négociateurs dans des boites de nuits et à d'autres activités illégales pour faliciter la signature des contrats.
%TODO diagramme MIE
