\subsection{Analyse stratégique de Probespoke}
\paragraph{Contexte du sur mesure}
 Plusieurs zones économiques se partagent le sur mesure. On peut classer ces zones en deux groupes. Le premier est celui des pays qui ont une forte tradition dans ces domaines à savoir l'Angleterre, la France et l'Italie. Ils se spécialisent dans le sur mesure haut de gamme à des prix atteignant souvent 1000 euros pour un costume complet. Le deuxième groupe est composé des pays ayant développé plus tardivement une industrie textile dans ce domaine. Il est notamment composé de la Chine, la Thailande, l'Europe de l'Est et le Maghreb. Cette zone produit des costumes de basse et moyenne qualités entre 200 euros et 700 euros. Les équipes sont néanmoins souvent formées par des maitre-tailleurs européens.  Les principaux importateurs sont les États-Unis, l'Europe et plus récemment la classe moyenne asiatique qui s'enrichit.
\paragraph{Avantages de la Thailande} Quels sont les avantages (et les inconvénients) de s'implanter en Thaïlande plutôt que dans d'autres pays du même groupe ? En Thaïlande la main d'œuvre est moins chère qu'au Maghreb et qu'en Europe de l'Est mais elle l'est plus qu'en Chine. Cependant le développement du sur mesure en Chine est plus récent et leur organisation n'est pas encore mature. Ainsi de multiples étapes de la confection sont sous-traités ce qui augmente les erreurs de communication. Ainsi de nombreux revendeurs attirées tout d'abord par les prix chinois reviennent ensuite en Thaïlande. Ce mouvement s'est accéléré cet été, en effet le désaccord commercial sino-américain pousse un grand nombre de tailleurs américains à quitter la Chine. L'Europe est un marché plus difficile à atteindre ; en effet les taxes d'import rendent l'Europe de l'Est et le Maghreb (qui profite d'accord commerciaux) compétitifs devant les prix thaïlandais. Finalement un atout de la Thaïlande est son industrie du divertissement. En effet dans de nombreuses entreprises thaïlandaises il existe un \textit{entertaiment officier} chargé d'amener les négociateurs dans des boites de nuits et à d'autres activités illégales pour faciliter la signature des contrats.

\paragraph{Avantage par rapport aux autres traders}
La plus-value délivrée par ProBespoke vient de deux facteurs : la simplification apportée aux tailleurs occidentaux et la différence du coup de main d'œuvre.
De nombreux autres commerciaux ont eu idée de profiter des avantages de la Thaïlande. Comment ProBespoke se démarque-t-il d’eux ? Premièrement le fondateur de ProBespoke est franco-thaï et a donc déjà des contacts sur place et une meilleure compréhension de l'économie thaïlandaise. C'est la première barrière à l'entrée dans ce secteur. De plus pour baisser encore les coûts, ProBespoke s'appuie sur un logiciel de gestion d'entreprise lui permettent de simplifier la comptabilité, le suivi des commandes et même les démarchages commerciaux. Finalement ProBespoke renvoie une image de tailleur 2.0 grâce à son application de prise de mesure qui lui permet de standardiser ses échanges avec les tailleurs partenaires.

Ces divers éléments sont résumé dans l'annexe deux, \textit{Matrice SWOT}
