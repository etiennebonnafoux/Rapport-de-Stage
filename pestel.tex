\subsection{Étude stratégique de ProBespoke}
Politiques : stabilité gouvernementale, politique fiscale, protection sociale, commerce extérieur, etc.
Économiques : cycle économique, évolution du PNB, taux d'intérêt, politique monétaire, inflation, chômage, revenu disponible, etc.
Sociologique : démographie, distribution des revenus, mobilité sociale, consumérisme, niveau d'éducation, attitude de loisir et de travail, etc.
Technologiques : dépenses publiques en R&D, investissements privés sur la technologie, nouveaux brevets ou découvertes, vitesse de transfert technologique, taux d'obsolescence, etc.
Écologiques : lois sur la protection de l'environnement, retraitement des déchets, consommation d'énergie, etc.
Légaux : lois sur les monopoles, droit du travail, législation sur la santé, normes de sécurité, etc.

L'entreprise ProBespoke évolue dans un milieu changeant rapidement et sa petite taille ne lui laisse que peu de marge de maneuvre.

Tout d'abord le protectionnisme économique naissant dans les pays occidentaux et surtout aux États-Unies pourait être une menace. Hors à y regarder de plus près les mesures pénalisent surtout ses concurrents directs chinois. Une autre menace politique serait celle d'en changement de régime en Thaïlande qui reste un pays mouvementé.

Ces divers éléments sont résumé dans l'annexe deux, \textit{Matrice SWOT}
