\section{Enrichissements personnels}
\subsection{Commentaire sur l'activité professionnel}

\begin{wrapfigure}[24]{o}{8cm}
\includegraphics[width=8cm]{image/bleuciel.jpg}
\end{wrapfigure}
\paragraph{Les outils de travail collaboratif}
Ce stage m'a permis de me former à des méthodes de travail efficace et sûr. Tout d'abord j'aimerais parler des outils de travail collaboratif qui ont été mis en œuvre durant ce stage. Le premier et le plus connus est le service de stockage en ligne Google Drive. Lors de notre arrivé, notre tuteur nous a partagé deux dossiers contenant les informations nécessaires pour le stage (mot de passe, code source des applications). Mais ces informations était mal rangée, dédoublé et avait souvent des noms peu explicite (par exemple : "mdp1,password") et cela nous a fait perdre un temps considérable. Je pense donc qu'une classification stricte des ressources et une nomenclature fixée et connue de tous les acteurs sont nécessaire au bon fonctionnement d'une entreprise. J'ai donc utilisé des outils plus professionnels tel Github pour gérer les version du code source des applications et laisser une trace intelligible des modifications apportées à mes potentiels successeurs. Cependant la version privée de Github étant payante, se ne peut pas être une solution à long terme pour ProBespoke.

\paragraph{Une communication efficace}
J'aimerais également mettre le doigt sur un aspect que je pense discriminant dans la conduite d'un projet réussi, la communication. Nous avons vécu durant notre stage deux sortes de périodes. La première était quand notre employeur était en Thaïlande ; nous avions alors l'habitude de travailler seul le matin, puis il nous rejoignait pour discuter des avancés entre 15h et 19h. L'autre période était ses voyages en France. Nous avions alors qu'une heure de discussion chacun au téléphone par jour. La vitesse de progression des différents projets s'en est fortement ressentit. Ainsi une mise au point régulière et efficace est selon moi une aide pour l’avancer d'un projet. Une entreprise souffre rarement d'un surplus de communication, en revanche l'inverse peut se produire. Pour nuancer, il faudrait ajouter que la longueur de l'échange n'est pas forcément proportionnelle à la quantité d'information échangé. En effet souvent nos discussions revenaient sur des points déjà présenté et se perdais en longueur.

\paragraph{Definir son domaine d'activité}
Une autre leçon importante que je retire de ce stage est qu'il faut connaitre sa place. En effet il faut pouvoir annoncer sans modestie ce que l'on est capable de faire, et d'un autre côté ne pas se montrer trop téméraire en annonçant des choses hors de nos disponibilités. Notre employeur qui prend des mesures dans différentes écoles, d'ingénieur et de commerce, nous a fait remarquer que la plus grande faiblesse des X est qu'ils ne savent pas se vendre. En effet peu d'entre nous comprennent à quel point leurs compétences peuvent être rare et recherchée ; ils ne les mettent donc pas en avant. D'un autre côté, il faut connaitre ces limites. J'ai notamment dû convaincre notre employeur de nous laisser quelque jour de recherche avant de se lancer dans une tâche critique, la migration d'un serveur. Effectivement il n'évalué pas bien, à mon avis, le risque des différentes opérations et la complexité qu'il y aurait à réparer la situation.

\paragraph{Définir les besoins du clients}
Finalement une autre composante critique dans une opération est de bien définir un cahier des charges. Cela peut être d'autant plus difficile que le client maitrise peu le langage technique voir qu'il ne connaisse même pas la possibilité de certaines options. Il faut donc construire avec lui une définition précise de son besoin. Plusieurs outils existent notamment les diagrammes de séquences\footnote{fr.wikipedia.org/wiki/Diagramme\_de\_sequence}  ou bien l'\textit{Unified Modeling Language} \footnote{openclassrooms.com/fr/courses/2035826-debutez-lanalyse-logicielle-avec-uml/2035851-uml-c-est-quoi}. Durant mon stage peu d'entre eux ont été mis en œuvre. Heureusement des rencontres ponctuelles avec notre employeur permettent de rectifier le cap régulièrement, mais sans cela le projet aurait bien pu ne pas aboutir.
