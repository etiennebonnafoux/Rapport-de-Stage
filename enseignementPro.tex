\section{Enrichissements personnels}
\subsection{Commentaire sur l'activité professionel}

\begin{wrapfigure}[24]{o}{8cm}
\includegraphics[width=8cm]{image/bleuciel.jpg}
\end{wrapfigure}
\paragraph{Les outils de travail collaboratif}
Ce stage m'a permis de me former à des méthodes de travail efficace et sûr. Tout d'abord j'aimerais parler des outils de travail collaboratif qui ont été mis en oeuvre durant ce stage. Le premier et le plus connus est le service de stoquage en ligne Google Drive. Lors de notre arrivé, notre tuteur nous a partagé deux dossiers contenant les informations nécessaire pour le stage (mot de passe, code source des applications). Mais ces informations était mal rangée, dédoublé et avait souvent des noms peux explicite (par exemple : "mdp1,password") et celà nous a fait perdre un temps considérable. Je pense donc qu'une classification stricte des resssource et une nomenclature fixée et connue de tout les acteurs sont nécessaire au bon fonctionnement d'une entreprise. J'ai donc utilisé des outils plus professionnel tel Github pour versionner le code source des apllications et laisser une traçe intélligible des modifications apportées à mes potnetiels successeurs. Cependant la version privé de Github étant payante, se ne peut pas être une solution à long terme pour Probespoke.

\paragraph{Une communication efficace}
J'aimerais également mettre le doigt sur un aspect que je pense discrimant dans la conduite d'un projet réussi, la communication. Nous avons vécus durant notre stage deux sortes de périodes. La première était quand notre employeur était en Thailande; nous avions alors l'habitude de travailler seul le matin, puis il nous rejoignais pour discuter des avancés entre 15h et 19h. L'autre période était ses voyages en France. Nous avions alors qu'une heure de discussion chacun au téléphone par jour. La vitesse de progression des différents projets s'en est fortement ressentit. Ainsi une mise au point régulière et efficace est selon moi une aide pour l'avancé d'un projet. Une entreprise souffre rarement d'un surplus de communication, en revanche l'inverse peut se produire. Pour nuancer, il faudrait ajouter que la longueur de l'échange n'est pas forcément proportionnel à la quantité d'information échangé. En effet souvent nos discussion revenais sur des points déjà présenté et se perdais en longueur.

\paragraph{Definir son domaine d'activité}
Un autre leçon importante que je retire de se stage est qu'il faut connaitre sa place. En effet il faut pouvoir annoncer sans modestie ce que l'on est capable de faire, et d'un autre côté ne pas se montrer trops téméraire en annoncant des choses hors de nos disponibilités. Notre employeur qui prend des mesures dans différentes écoles, d'ingénieur et de commerce, nous à fait remarquer que la plus grande faiblesse des X est qu'ils ne savent pas se vendre. En effet peu d'entre nous savent à quel point leurs compétences peuvent être rare et recherchée; ils ne les mettent donc pas en avant. D'un autre côté, il faut connaitre ces limites. J'ai notemment dû convaincre notre employeur de nous laisser quelque jour de recherche avant de se lancer dans une tâche critique, la migration d'un serveur. En effet il n'évalué pas bien, à mon avi, le risque de la chose et la difficulté qu'il y aurait à réparer la situation.

\paragraph{Definir les besoins du clients}
Finalement une autre composante critique dans une opération est de bien définir un cahier des charges. Celà peut être d'autant plus difficile que le client maitrise peu le language technique ou bien qu'il ne connaisse même pas que certaine options existent. Il faut donc construire avec lui une définission précise de son besoin. Plusieurs outils existent notemment les diagrammes de séquences\footnote{fr.wikipedia.org/wiki/Diagramme\_de\_sequence}  ou bien l'\textit{Unified Modeling Language} \footnote{openclassrooms.com/fr/courses/2035826-debutez-lanalyse-logicielle-avec-uml/2035851-uml-c-est-quoi}. Durant mon stage peu d'entre eux ont été mis en oeuvre. Heureusement des rencontres régulière avec notre employeur permet de réctifier le cap régulièrement, mais sans celà le projet aurait bien pus ne pas aboutir.
