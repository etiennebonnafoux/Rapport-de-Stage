
%TODO a faire mieux
\paragraph{ERP : }
\begin{wrapfigure}[4]{o}{4cm}
\qrcode[height=1in]{https://en.wikipedia.org/wiki/Enterprise_resource_planning}
\end{wrapfigure}
\paragraph{}

\textbf{E}nterprise \textbf{R}esource \textbf{P}lanning  en anglais. Outil informatique permettant de mettre en lien différente partie d'une même entreprise. Cet outil doit assurer l'intégrité des ressources ainsi qu'il doit premettre de faire des audits.

\paragraph{Odoo : }
\begin{wrapfigure}[5]{i}{4cm}
\qrcode[height=1in]{https://www.odoo.com/fr_FR/}
\end{wrapfigure}
\paragraph{}
Odoo est un logiciel de gestion d'entreprise dont la première version est sortie en 2004. Il comprend plusieurs modules dont la gestion des ventes, des ressources humaines ou de la comptabilité. Si ça version basique est gratuite il existe une version payante incluant un support.

\paragraph{UML : }
\begin{wrapfigure}[5]{o}{4cm}
\qrcode[height=1in]{https://en.wikipedia.org/wiki/Unified_Modeling_Language}
\end{wrapfigure}
\paragraph{}
\textbf{U}nified \textbf{M}odeling \textbf{L}anguage en anglaise. Language et suite de diagramme permettant de définir un cahier des charges dans le domaine informatique. L'analyse se fait d'abord du point de vue du client puis déscend au différentes solutions techniques envisagées et au flux d'information.
